\section{INTRODUCTION}

Machine learning techniques, coupled with data, are used to solve a multitude of high-dimensional problems with great success, such as those in the area of image recognition. For instance, image recognition is employed in self-driving cars to navigate on the roads. However, research has shown that these machine learning models are not robust to adversarial attacks and can be exploited by injecting specifically designed samples to training data or by creating test samples based on the decision boundary of the algorithm to misguide the classification result. For example, Papernot et. al. showed that it is possible to craft an image that would appear to be a stop sign but would be classified as a yield sign by some class of deep neural networks \cite{papernot1}. Furthermore, Papernot et. al. also found that the perturbation technique they used to construct such adversarial samples is applicable to a variety of other classifiers, such as support vector machines and logistic regression \cite{papernot3}. These findings demonstrated that machine learning systems are susceptible to malicious attacks. One such example would be to alter the image of road signs received by autonomous driving systems in order to manipulate the behaviour of target vehicles, which could lead to dire consequences. Thus, understanding the vulnerabilities of machine learning systems and the methods to exploit them is crucial for application of machine learning in practical settings.